\documentclass[helvetica,openbib,totpages]{europecv}
\usepackage[T1]{fontenc}
\usepackage{graphicx}
\usepackage[a4paper,top=1.27cm,left=1cm,right=1cm,bottom=2cm]{geometry}
\usepackage[english]{babel}
\usepackage{bibentry}
\usepackage{hyperref}
\usepackage{xcolor}

\renewcommand{\ttdefault}{phv} % Uses Helvetica instead of fixed width font

\ecvname{Sabatini, Paolo}
\ecvfootername{Paolo Sabatini}
\ecvaddress{40, via T. Sabatini, 53021, Abbadia San Salvatore (IT)\newline 60, Chemin des Collines, 01360, Sergy (FR)\newline 134, Hannoverschestrasse, 37077, G\"ottingen (DE)}
\ecvtelephone{+393385968853}
\ecvemail{\href{mailto:paolosbtn@gmail.com}{\color{blue!80!black}paolosbtn@gmail.com}}
\ecvnationality{Italian}
\ecvdateofbirth{Abbadia San Salvatore}
\ecvdateofbirth{09/06/1991}
\ecvgender{Male}




\ecvfootnote{For more information go to \url{http://europass.cedefop.eu.int}\\
\textcopyright~European Communities, 2003.}

\begin{document}
\selectlanguage{english}


\begin{europecv}
\ecvpersonalinfo[5pt]

\ecvsection{Work experiences}
\ecvitem{2016\,-\,}{\textbf{PhD in Physics}}
\ecvitem{\small Principal subjects/Occupational skills covered}{\small{Currently I'm involved in the GAUSS doctoral program at 2. Physikalisches Institut at the University of G\"ottingen (DE), working for the ATLAS experiment collaboration at CERN (European Organization for Nuclear Research).\vspace{1mm}

The first year has been dedicated to the achievement of the authorship qualification, accomplished in November 2017. The qualification project concerned studies on the calibration for the innermost layer in the ATLAS Inner Detector by analysing collision and module production data, and by recording new data in dedicated laboratory measurement. \vspace{1mm}

After the qualification as an ATLAS author, I moved to CERN to start the main topic of the PhD project regarding the measurement of the four-top quark production process in proton-proton collision at $13$ TeV of centre-of-mass energy. I'm currently working on this topic, being one of the main analyser of the research group (around a dozen of active people). In the meantime I continued helping out in studies for the Inner Detector and MonteCarlo samples generation, and actively participated to the experiment data acquisition. \vspace{1mm}

This very broad activity provided to me experience in coding and framework building in several languages, version control platform use, data analysis instruments e.g. Multi-Variate-Analysis techniques and fitting alghoritms. Moreover, from hardware activity, I got experience in semiconductor physics and its applications such as silicon-pixel detectors. Being at CERN led me networking with several different research groups, stimulating new interests and collaborating in short-term projects.
The PhD is expected to finish by the end of 2019.}}
\ecvitem{\small Name and type of organization providing education and training}{\small{Georg-August-Universit\"at G\"ottingen (DE)}}\\

\ecvsection{Education and training}
\ecvitem{2013\,-\,2016}{\textbf{MSc in Physics} (Curriculum Particle Physics)}
\ecvitem{\small Title of qualification awarded}{\small{Degree in Physics from the 30/05/2016 by a vote of 110/110 cum laude.}}
\ecvitem{\small MSc thesis}{\small{\emph{First level triggering at NA62 rare kaon decay experiment}}}
\ecvitem{\small Principal subjects/Occupational skills covered}{\small{The MSc course covered several different scientific fields such as General Relativity, Quantum Field Theory and Particle Physics, Standard Model, Montecarlo Methods, Data Analysis and Astrophysics.\vspace{1mm}

All the exams in the curriculum have been passed by June 2015. Since April 2015 I worked on the master thesis in collaboration with the NA62 Research Group at INFN (National Institute of Nuclear Physics). The thesis work (available in the \href{http://etd.adm.unipi.it/ETD-db/ETD-relatori/assign_session?username=465219&password=PK2ZhVGYP}{\color{blue!80!black}ETD system}) concerns the implementation of the simulation for the level-0 trigger at NA62 experiment at CERN (Cente Nucleare de Recherce Nucleare). The simulation helped the optimisation of the trigger conditions to increase signal efficiency in data acquisition, representing an important tool for testing the hardware system performance.}}
\ecvitem{\small Name and type of organization providing education and training}{\small{Università di Pisa (IT).}}\\

\ecvitem{2010\,-\,2013}{\textbf{BSc in Physics}}
\ecvitem{\small Title of qualification awarded}{\small{Degree in Physics from the 26/09/2013 by a vote of 107/110.}}
\ecvitem{\small Principal subjects/Occupational skills covered}{\small{The covered subjects are available at \href{http://www.df.unipi.it/didatticanuova/1011/descrizione-0}{\color{blue!80!black}http://www.df.unipi.it/didatticanuova/1011/descrizione-0}. Additional courses in "Analytical Mechanics" and "Partial Differential Equations" have been attended.
Final dissertation on \emph{"The EPR paradox of Quantum Mechanics"}. Supervisor: Prof. G. Paffuti.}}
\ecvitem{\small Name and type of organization providing education and training}{\small{Università di Pisa (IT)}}\\

\ecvitem{2005\,-\,2010}{\textbf{High School (scientific/technological curriculum)}}
\ecvitem{\small Title of qualification awarded}{\small{High School Diploma with a vote of 100/100 cum laude.}}
\ecvitem{\small Name and type of organization providing education and training}{\small{Liceo Scientifico Tecnologico "Amedeo Avogadro", Abbadia San Salvatore (IT).}}

\ecvsection{Additional Schooling}
\ecvitem{2018}{\textbf{Deutsche Physikalische Gesellschaft (DPG) Frühjahrstagung}, W\"urzburg.\newline
{\small Abstract of my talk available at this \href{https://www.dpg-verhandlungen.de/year/2018/conference/wuerzburg/part/t/session/42/contribution/10}{\color{blue!80!black}link}}.}
\ecvitem{2017}{\textbf{Deutsche Physikalische Gesellschaft (DPG) Frühjahrstagung}, M\"unster.\newline
{\small Abstract of my talk available at this \href{https://www.dpg-verhandlungen.de/year/2017/conference/muenster/part/t/session/94/contribution/}{\color{blue!80!black}link}}.}
\ecvitem{2015}{\textbf{Experience at NA62 experiment}, CERN (CH).}
\ecvitem{2014}{\textbf{HASCO Summer School}, University of G\"ottingen (DE).

{\small Final dissertation on \emph{"The quest of quark-gluon plasma"}. Co-author: Rocìo Saéz Blàzquez. Supervisor: Prof. Rosario Nania.}}
\ecvitem{2009}{\textbf{Orientation Courses of Scuola Normale Superiore of Pisa}, San Miniato (IT).}
\ecvitem{2008}{\textbf{ECDL}, European Computer Driving Licence.}

\ecvsection{Personal skills and~competences}

\ecvmothertongue[5pt]{Italian}
\ecvitem{\large Other language(s)}{\begin{description}
 \item[English]{\small Achieved the Preliminary English Test Certificate (PET) with merit, released by Cambridge University, in 24/07/2009 by a vote of 85/100. Speaking in English on a daily-basis since 2016.}
  \item[German]{\small Achieved the B1 level during my stay in G\"ottingen in 2017 by attending language courses at university.}
  \item[French]{\small Learned during three years in the secondary schools and improved during my stay at CERN.}
  
\end{description}}
\ecvlanguageheader{(*)}
\ecvlanguage{English}{C1}{C1}{C1}{C1}{C1}
\ecvlanguage{German}{B1}{B1}{B1}{B1}{B1}
\ecvlanguage{French}{A2}{A2}{A2}{A2}{A2}
\ecvlanguagefooter[10pt]{(*)}

\ecvitem[10pt]{\large Computer skills and competences}{\small Use of Unix and Windows operating systems. Experienced in C/C++, phyton, bash, used during my research. Good experience in version-control platform, such as svn or GitHub. Some experience accumulated in HTML and Qt for private interests. Use of \LaTeX\,\,\,for document and slides realisation. Good experience in Office package. Broad use of ROOT software.}

\ecvsection{Additional information}
\ecvitem{2010}{Member of the \textbf{National Excellence Register}.}

\ecvsection{Profiles}
\ecvitem{GitHub}{\href{https://github.com/paolosabatini}{\color{blue!80!black}https://github.com/paolosabatini}}
\ecvitem{CERN Phonebook}{\href{https://phonebook.cern.ch/phonebook/\#personDetails/?id=771462}{\color{blue!80!black}https://phonebook.cern.ch/phonebook/\# personDetails/?id=771462}}
\ecvitem{Webpage}{\href{http://psabatin.web.cern.ch/psabatin/index.html}{\color{blue!80!black}http://psabatin.web.cern.ch/psabatin/index.html}}
\end{europecv}


\end{document} 
